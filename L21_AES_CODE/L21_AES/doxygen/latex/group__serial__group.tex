\hypertarget{group__serial__group}{}\section{Serial Interface (Serial)}
\label{group__serial__group}\index{Serial Interface (\+Serial)@{Serial Interface (\+Serial)}}
See \hyperlink{serial_quickstart}{Quick start guide for Serial Interface service}.

This is the common A\+P\+I for serial interface. Additional features are available in the documentation of the specific modules.\hypertarget{group__serial__group_serial_group_platform}{}\subsection{Platform Dependencies}\label{group__serial__group_serial_group_platform}
The serial A\+P\+I is partially chip-\/ or platform-\/specific. While all platforms provide mostly the same functionality, there are some variations around how different bus types and clock tree structures are handled.

The following functions are available on all platforms, but there may be variations in the function signature (i.\+e. parameters) and behaviour. These functions are typically called by platform-\/specific parts of drivers, and applications that aren't intended to be portable\+:
\begin{DoxyItemize}
\item usart\+\_\+serial\+\_\+init()
\item usart\+\_\+serial\+\_\+putchar()
\item usart\+\_\+serial\+\_\+getchar()
\item usart\+\_\+serial\+\_\+write\+\_\+packet()
\item usart\+\_\+serial\+\_\+read\+\_\+packet() 
\end{DoxyItemize}